\documentclass{article}
\usepackage{amsmath}
\usepackage{amssymb}

\begin{document}

\section{Page 1}

\subsection{Question 1:
    What is a micro-interaction? Name any two of the four components of a
    micro-interaction.
    Give one example of a micro-interaction in an interface.}

A micro-interaction is a small, contained product moment that revolves around a single use case.
The four components of a micro-interaction are: trigger, rules, feedback, and loops.
An example of a micro-interaction is the heart icon on Instagram that changes color when you double tap it.

\subsection{Question 2:
    What are the problems with using a red circle to indicate “stop”, which
    changes to a green circle when it is safe for the user to proceed with their
    action?}

The problem with using a red circle to indicate “stop” is that it is not universally understood.
In some cultures, red is associated with danger, while in others, it is associated with luck.
This can lead to confusion and misinterpretation of the message.
Additionally, color blindness can make it difficult for some users to distinguish between red and green, further complicating the message.
The use of color alone to convey meaning can also be problematic, as it relies on the user’s ability to perceive color accurately.
This can be challenging for users with visual impairments or in situations where the lighting is poor.

\subsection{What is the difference between menu depth and menu breadth? Which is
    better to use to present menu choices to users, and why?}

Menu depth refers to the number of levels in a menu hierarchy, while menu breadth refers to the number of choices at each level.
A shallow menu with a large breadth is generally better for presenting menu choices to users, as it allows users to quickly access the information they need without having to navigate through multiple levels of the menu.
A deep menu with a small breadth can be overwhelming and confusing for users, as it requires them to remember the hierarchy and navigate through multiple levels to find the desired information.
A shallow menu with a large breadth is more user-friendly and intuitive, as it minimizes the cognitive load on users and allows them to easily find the information they are looking for.

\subsection{With regard to human memory, what is a retrieval cue? Give two examples
    of retrieval cues used in interface design.}

A retrieval cue is a stimulus that helps trigger the recall of information stored in memory.
Two examples of retrieval cues used in interface design are:
1. Icons: Icons are visual cues that represent specific actions or concepts.
For example, a magnifying glass icon is commonly used to represent the search function.
2. Color coding: Color coding is a visual cue that uses different colors to represent different categories or types of information.
For example, red is often used to indicate errors or warnings, while green is used to indicate success or completion.

\newpage

\section{Page 2}

\subsection{(a) One method of preventing errors in design is to use a forcing function.
    Forcing functions can be especially useful in safety-critical systems.
    In interface design, we distinguish between three types of forcing functions.
    For each of the following scenarios, identify and name the type of forcing
    function being used.}

\subsubsection{(i) Dialog window appears asking if you want to save your work before
    closing a document with unsaved changes.}

The type of forcing function being used in this scenario is a lock-in forcing function.
A lock-in forcing function prevents the user from taking an action until a specific condition is met.
In this case, the user is prevented from closing the document until they have saved their work.

\subsubsection{(ii) ATM forces you to take your card first before releasing your money.}

The type of forcing function being used in this scenario is an interlock forcing function.
An interlock forcing function prevents the user from taking an action until a specific condition is met.
In this case, the user is prevented from taking their money until they have taken their card.

\subsubsection{(iii) When paying for an item purchased online, you cannot complete the
    payment until you’ve entered the one-time security code}

The type of forcing function being used in this scenario is a lock-out forcing function.
A lock-out forcing function prevents the user from taking an action until a specific condition is met.
In this case, the user is prevented from completing the payment until they have entered the one-time security code.

\subsection{(b) We distinguish between two types of user errors: slips and mistakes. Slips
    are unconscious errors – right intention, but wrong action.
    In no more than one sentence, identify which type of user is more prone to
    “slips”, and why}

Users who are more experienced with a system are more prone to slips, as they may rely on automatic or habitual actions rather than conscious thought when interacting with the system.
The familiarity with the system can lead to slips, as users may perform the wrong action unintentionally due to muscle memory or routine.
We distinguish between two types of user errors: slips and mistakes. Slips are unconscious errors – right intention, but wrong action.
In no more than one sentence, identify which type of user is more prone to “slips”, and why.
Another example of a slip is when a user types the wrong password due to muscle memory or habit, even though they know the correct password.
On the other hand, mistakes are conscious errors – wrong intention, wrong action.
\end{document}